\section{Methodology}

\subsection{Data Sources}

Our analysis leverages a comprehensive dataset drawn from multiple sources to encapsulate Yemen's complex market dynamics:

\begin{itemize}
    \item \textbf{Commodity Prices}: Weekly prices for key staples—wheat, rice, and sugar—collected from multiple Yemeni markets by the World Food Programme (WFP). The high-frequency nature of these data permits detailed tracking of price movements across different political control zones.

    \item \textbf{Exchange Rates}: Official and parallel market exchange rates provided by the Central Bank of Yemen, as referenced in recent World Bank reports.\autocite{worldbank2022} Including both exchange rate measures is crucial to capture distortions arising from the dual rate system. The differential between rates in Houthi-controlled versus government-controlled territories is explicitly incorporated into our econometric models.

    \item \textbf{Geographic Data}: Market locations and administrative boundaries sourced from the ACAPS Yemen Analysis Hub. These data enable the construction of spatial weight matrices and facilitate analysis of geographic price transmission.\autocite{anselin1988} We augment traditional distance metrics with information on political control boundaries to capture how fragmentation affects spatial market relationships.

    \item \textbf{Conflict Data}: Security incidents and measures of conflict intensity from the Armed Conflict Location \& Event Data Project (ACLED). This information enables us to assess how conflict conditions influence market integration and price transmission.\autocite{mansour2021} We create conflict intensity indices that vary both temporally and spatially to capture the heterogeneous impact of violence on market functioning.
\end{itemize}

\subsection{Econometric Framework}

Our primary econometric strategy centers on threshold cointegration models to capture nonlinearities in price transmission attributable to conflict-induced transaction costs.

\subsubsection{Unit Root and Cointegration Testing}

We begin by assessing the integration properties of the commodity price and exchange rate series using the Augmented Dickey-Fuller Generalized Least Squares (ADF-GLS) test, which offers improved power in small samples.\autocite{elliott1996} Complementary tests (Phillips-Perron and KPSS) are used to validate these results. For long-run equilibrium relationships, the Johansen cointegration test is applied, with adjustments for structural breaks via the Gregory-Hansen test where necessary.\autocite{gregory1996}

The ADF-GLS test specification for a series $y_t$ is:
\begin{equation}
\Delta y_t^d = \alpha y_{t-1}^d + \sum_{i=1}^p \beta_i \Delta y_{t-i}^d + \epsilon_t
\end{equation}

where $y_t^d$ is the locally detrended series. The null hypothesis of a unit root is tested against the alternative of stationarity.

Given Yemen's volatile economic environment, we also apply the Zivot-Andrews test to account for potential structural breaks in the price series, particularly around significant conflict events or policy changes such as the 2017 Central Bank decision to float the rial.\autocite{zivot1992}

\subsubsection{Threshold Estimation and Testing}

To detect nonlinear adjustment regimes, we apply the Tsay test for threshold nonlinearity.\autocite{tsay1989} This test identifies whether price adjustments occur only beyond a certain threshold, which in our context represents transaction costs such as security fees and logistical barriers. When significant threshold effects are detected, we estimate a Threshold Vector Error-Correction Model (TVECM) following Hansen and Seo's methodology.\autocite{hansen2002}

The TVECM specification for two price series $P_{1t}$ and $P_{2t}$ is:

\begin{equation}
\Delta P_t = 
\begin{cases}
A_1 X_t(\beta) + u_{1t} & \text{if } w_t(\beta) \leq \gamma \\
A_2 X_t(\beta) + u_{2t} & \text{if } w_t(\beta) > \gamma
\end{cases}
\end{equation}

where $P_t = (P_{1t}, P_{2t})'$, $w_t(\beta) = P_{1t-1} - \beta P_{2t-1}$ is the error correction term, $\gamma$ is the threshold parameter, and $X_t(\beta)$ includes lags of $\Delta P_t$ and the error correction term.

The threshold parameter $\gamma$ is estimated using Hansen and Seo's maximum likelihood approach, with a grid search over potential threshold values. We test the significance of the threshold effect using the sup-LM test, which compares the threshold model against a linear VECM alternative.

\subsubsection{Asymmetric Adjustment Analysis}

Recognizing that price adjustments may exhibit asymmetry—where, for instance, price spikes are corrected more rapidly than price declines—we employ the M-TAR model of Enders and Siklos.\autocite{enders2001} This allows us to capture directional differences in the error-correction process, which is particularly relevant in markets subject to sudden shocks from conflict-related disruptions.

The M-TAR specification is:

\begin{equation}
\Delta \hat{\mu}_t = I_t \rho_1 \hat{\mu}_{t-1} + (1-I_t) \rho_2 \hat{\mu}_{t-1} + \sum_{i=1}^p \gamma_i \Delta \hat{\mu}_{t-i} + \epsilon_t
\end{equation}

where $\hat{\mu}_t$ is the estimated residual from the cointegrating relationship, and $I_t$ is the Heaviside indicator function:

\begin{equation}
I_t = 
\begin{cases}
1 & \text{if } \Delta \hat{\mu}_{t-1} \geq \tau \\
0 & \text{if } \Delta \hat{\mu}_{t-1} < \tau
\end{cases}
\end{equation}

with $\tau$ being the threshold value. We test the null hypothesis of symmetric adjustment ($\rho_1 = \rho_2$) against the alternative of asymmetric adjustment.

\subsubsection{Spatial Econometric Analysis}

Given Yemen's fragmented geography, spatial econometric techniques are used to analyze geographic price transmission. Moran's I statistic is first computed to test for spatial autocorrelation among VECM residuals. Where significant spatial dependence is detected, we estimate spatial lag models to quantify the influence of neighboring markets on local prices.\autocite{anselin1988}

The spatial lag model is specified as:

\begin{equation}
P_t = \rho W P_t + X_t \beta + \epsilon_t
\end{equation}

where $P_t$ is a vector of prices across markets, $W$ is a spatial weight matrix based on market connectivity (adjusted for conflict barriers), $\rho$ is the spatial autoregressive parameter, $X_t$ are control variables including exchange rate differentials, and $\epsilon_t$ is the error term.

The weight matrix $W$ is constructed using geographic distances between markets, modified to account for conflict intensity along transport routes. This approach allows us to capture how political fragmentation alters the effective economic distance between markets, even when geographic proximity would suggest stronger integration.

\subsection{Simulation of Market Integration Scenarios}

To evaluate the potential impact of policy interventions, particularly exchange rate unification, we simulate alternative market integration scenarios:

\subsubsection{Exchange Rate Unification Simulation}

Within the VECM/TAR framework, we incorporate the dual exchange rate differential as an exogenous variable. By setting this differential to zero, we simulate a unified exchange rate environment. This scenario analysis allows us to observe changes in the speed of price adjustment and the reduction in the threshold parameter, reflecting a lower arbitrage barrier.

Specifically, we re-estimate the TVECM with the constraint that $E_{S,t} = E_{A,t}$, where $E_{S,t}$ is the exchange rate in Sana'a and $E_{A,t}$ is the exchange rate in Aden. The resulting changes in the threshold parameter $\gamma$ and the adjustment coefficients provide quantitative insights into how exchange rate unification might enhance market integration by narrowing the ``no-arbitrage'' band.

\subsubsection{Spatial Connectivity Simulation}

In parallel, we adjust the spatial weight matrix to simulate improved market connectivity—reflecting scenarios such as the reopening of key trade corridors or improved security conditions. By reducing the effective distance between markets, the spatial lag model forecasts a decrease in price dispersion. These simulations provide policymakers with quantitative insights into the benefits of enhancing geographic connectivity.

\subsection{Robustness and Diagnostic Tests}

To ensure the reliability of our findings, we implement several robustness and diagnostic checks:

\subsubsection{Structural Break Tests}

Beyond the Gregory-Hansen test, we employ Bai-Perron multiple break tests to identify shifts in the cointegration relationship corresponding to major conflict events or policy changes.\autocite{bai1998} This approach recognizes that market relationships in Yemen may have fundamentally changed at various points during the conflict, requiring flexible modeling approaches.

\subsubsection{Residual Diagnostics}

We examine model residuals for serial correlation (using the Breusch-Godfrey LM test), heteroskedasticity (via White's test), and normality (using the Jarque-Bera test). In cases where serial correlation is detected, lag lengths are adjusted or Newey-West robust errors are employed to ensure valid inference.

\subsubsection{Sensitivity Analyses}

We re-estimate models under alternative specifications (e.g., symmetric ECMs, varied lag lengths based on AIC/BIC) and perform rolling estimations to assess the stability of our results. Additional analyses exclude extreme outliers—often associated with peak conflict incidents—to verify the robustness of the core findings.

This comprehensive methodological approach enables us to identify the key drivers of market fragmentation in Yemen and to quantify the potential impact of policy interventions on market integration. By combining threshold cointegration techniques with spatial econometric methods, we capture both the nonlinear nature of price adjustment processes and the geographic dimension of market relationships, providing a nuanced understanding of how conflict affects economic integration.