\section{Conclusion}

This paper has examined market integration dynamics in conflict-affected Yemen, with particular emphasis on how political fragmentation, security barriers, and dual exchange rate regimes impede price transmission across markets. Our econometric analysis, leveraging threshold cointegration and spatial econometric techniques, yields several important findings.

First, Yemen's commodity markets exhibit significant threshold effects, with price adjustments occurring only when differentials exceed substantial transaction costs associated with conflict conditions. The threshold parameters estimated for key staples range from 15\% to 28\%—far exceeding typical transport costs in integrated markets. These thresholds vary systematically with conflict intensity and political boundaries, providing quantitative evidence of how conflict disrupts market functioning.

Second, we find pronounced asymmetry in price adjustment processes, with upward price movements transmitting more rapidly than downward corrections. This asymmetry, combined with the high threshold parameters, helps explain the persistent inflation and food insecurity challenges observed in Yemen despite periodic improvements in supply conditions. Price signals that would normally coordinate market responses are significantly dampened by conflict-related barriers.

Third, our spatial analysis demonstrates that Yemen's markets have fragmented into distinct geographic clusters that align closely with political control boundaries rather than traditional economic regions. The dual exchange rate system emerges as a significant predictor of spatial price disparities, explaining an additional 37\% of variance in price gaps beyond what geographic and security factors account for.

Perhaps most importantly, our policy simulations indicate that exchange rate unification could substantially enhance market integration by reducing threshold parameters by 42\% and increasing adjustment speeds by 57\%. These improvements would lead to more efficient arbitrage, narrowing price gaps between regions and potentially improving food security outcomes. While physical market access improvements also yield significant benefits, monetary unification appears to offer the more substantial immediate gains, potentially serving as a foundational step toward broader economic reintegration.

These findings have important implications for international efforts to support economic stability in Yemen. Rather than focusing exclusively on reducing physical trade barriers, which may be difficult in active conflict zones, interventions that address the dual exchange rate system could provide significant economic benefits with potentially lower implementation barriers. Furthermore, our results suggest that partial improvements in market integration are achievable even before comprehensive peace is established, providing an economic pathway to support humanitarian outcomes.

The methodology developed in this paper also contributes to the broader literature on market integration in conflict settings by explicitly modeling the nonlinear and asymmetric price transmission processes characteristic of fragmented markets. By combining threshold cointegration techniques with spatial econometric methods, we provide a more comprehensive framework for understanding how conflict disrupts market relationships across both geographic and temporal dimensions.

Several limitations of this study should be acknowledged. First, data constraints in conflict zones limit the granularity of our analysis, particularly regarding internal transaction costs that cannot be directly observed. Second, our simulations necessarily simplify the complex political economy factors that would influence the implementation of exchange rate unification. Future research could address these limitations by incorporating additional microeconomic survey data on trader costs and developing more sophisticated political economy models of monetary policy in fragmented states.

Despite these limitations, our findings provide empirically grounded insights into the economic consequences of market fragmentation in Yemen and the potential pathways toward reintegration. As the international community continues to seek effective interventions in this protracted crisis, understanding the structural impediments to market functioning—and the potential leverage points for improvement—remains essential for designing policies that support economic resilience and humanitarian outcomes.