\section{Introduction}

Yemen's protracted conflict has not only devastated infrastructure and public institutions but has also severely fragmented its commodity markets. Since the escalation of hostilities in 2014, the country's economic landscape has been characterized by dual exchange rate regimes and significant disruptions in trade routes. In areas controlled by the internationally recognized government versus those under Houthi authority, divergent monetary policies and security constraints have created arbitrage opportunities and price distortions. The de facto dual economy—with separate rates as reported by the Central Bank of Yemen in Aden and the Houthi-controlled banking system in Sana'a—generates nonuniform price signals that fundamentally undermine the Law of One Price, a cornerstone principle of integrated markets.

Market integration is critical for economic stability, especially in fragile, conflict and violence situations (FCS). In Yemen, high transaction costs associated with blocked trade corridors and security barriers exacerbate market segmentation. Transaction costs in Yemen have become exceptionally burdensome, reflecting not only traditional elements such as transportation expenses but also conflict-specific costs including checkpoint fees, protection payments, and risk premiums for operating in insecure environments. These costs create substantial thresholds that must be exceeded before arbitrage becomes economically viable, resulting in persistent price differentials across regions.

The World Bank has documented that Yemen's GDP contracted by approximately 54\% in real per capita terms between 2015 and 2023, with continued negative growth projected for 2024.\autocite{worldbank2022} This economic deterioration has been accompanied by currency volatility, with the Yemeni rial experiencing divergent valuations—for example, in early 2023, exchange rates varied dramatically from approximately 600 rials per USD in Houthi-controlled territories to 1,225 rials per USD in government-controlled areas.\autocite{sanaa2023} These substantial discrepancies create significant market distortions that impede economic recovery and exacerbate food insecurity in a country where approximately 21.6 million people require humanitarian assistance.

By analyzing market fragmentation and exploring pathways toward reintegration through policy interventions such as exchange rate unification and improved connectivity, this paper aims to enhance understanding of economic resilience mechanisms in conflict zones. This research carries substantial policy relevance, particularly as international institutions seek effective interventions to support economic stability in FCAS environments. Furthermore, the paper addresses important theoretical questions regarding the applicability of traditional market integration models in conflict settings where normal arbitrage mechanisms are constrained.

This paper outlines both the theoretical and empirical foundations of market integration in conflict zones and proposes an econometric framework—centered on threshold cointegration models supplemented by spatial econometric techniques—to quantify and simulate the impact of policy reforms. By focusing specifically on commodity price transmission across Yemen's fragmented markets, we aim to identify effective policy levers for enhancing market integration, with particular emphasis on exchange rate unification as a potential catalyst for economic stabilization.