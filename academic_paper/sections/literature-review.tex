\section{Literature Review}

\subsection{Theoretical Foundations}

The Law of One Price (LOP) posits that in an integrated market, identical goods should command the same price when adjusted for transport and transaction costs. Early theoretical contributions established that arbitrage forces can eliminate price differentials up to a threshold defined by these costs.\autocite{baulch1997, ravallion1986} Classical arbitrage models assume continuous adjustment; however, when trade barriers exist—whether due to geography, infrastructural deficiencies, or political disruptions—these transaction costs become significant enough to sustain persistent price gaps.

Fackler and Goodwin provide a comprehensive framework for spatial price analysis, emphasizing that market integration exists along a continuum rather than as a binary state.\autocite{fackler2001} In fully integrated markets, price shocks transmit completely across regions, ensuring that spatial price differentials reflect only transaction costs. However, in partially integrated markets—common in developing countries and particularly prevalent in conflict settings—price transmission may be incomplete, asymmetric, or subject to substantial thresholds before arbitrage occurs.

The theoretical literature increasingly recognizes the nonlinear nature of market adjustment processes. Rather than continuous arbitrage in response to any price differential, Balke and Fomby introduced the concept of threshold cointegration, where price adjustments only occur once differentials exceed transaction costs.\autocite{balke1997} This ``band'' of inaction, where small price differences persist without triggering arbitrage, is particularly relevant to conflict-affected environments like Yemen, where security checkpoints, damaged infrastructure, and political barriers significantly elevate transaction costs.

\subsection{Empirical Evidence in Conflict-Affected Economies}

Empirical studies in conflict-affected regions consistently document that market integration is severely hampered by structural impediments. In countries such as Ethiopia, Somalia, and Syria, research has shown that conflict-induced disruptions lead to localized monopolistic practices and persistent price differentials.\autocite{dercon1995, little2007, worldbank2020} In Yemen specifically, World Bank analyses have documented how the coexistence of official and parallel exchange rates distorts price signals, as commodity markets operate under conditions that preclude full arbitrage.\autocite{worldbank2022}

Mansour et al. examined market integration in Syria during active conflict, finding that the destruction of trade routes and emergence of internal checkpoints significantly reduced the speed of price transmission between previously connected markets.\autocite{mansour2021} Their findings highlight how conflict not only increases transaction costs but also fundamentally alters the structure of market relationships, creating isolated price islands in areas where trade flows are severely constrained. This pattern appears consistent with Yemen's experience, where price differentials between Houthi-controlled and government-controlled territories persist beyond what traditional transport costs would justify.

Recent empirical work has increasingly turned to advanced econometric techniques to capture these nonlinear dynamics. Threshold cointegration models—first developed by Balke and Fomby and refined by Hansen and Seo—allow for the modeling of situations where price convergence only occurs once price gaps exceed a critical threshold.\autocite{balke1997, hansen2002} This ``band'' structure is particularly relevant to Yemen, where small differences in commodity prices persist due to high transaction costs (e.g., security risks, checkpoints), while larger deviations trigger arbitrage.

Moreover, Enders and Siklos's application of the Momentum Threshold Autoregressive (M-TAR) model to capture asymmetric adjustment dynamics has proven valuable for understanding how prices react differently to upward versus downward shocks in conflict zones.\autocite{enders2001} This approach recognizes that price increases (often driven by supply disruptions in conflict settings) may transmit differently than price decreases, with important implications for market integration analysis.

\subsection{Alternative Approaches}

Although threshold cointegration offers a robust framework, alternative methodologies provide complementary insights. Guney, Goodwin, and Riquelme introduced Generalized Additive Vector Autoregression (GAVAR) models, which allow for semiparametric modeling of gradual, continuous adjustments.\autocite{guney2019} This approach offers flexibility in capturing price dynamics when adjustment processes vary continuously with shock magnitude, rather than exhibiting discrete threshold effects.

Additionally, spatial econometric models as formalized by Anselin facilitate examination of geographic dependencies by incorporating spatial weight matrices to account for proximity effects.\autocite{anselin1988} These approaches are particularly relevant for Yemen, where geographic fragmentation and localized conflict intensity create spatially heterogeneous market conditions. By accounting for spatial autocorrelation in price movements, these models can capture how market integration varies across different regions of the country, providing granular insights beyond what time-series models alone can offer.

The literature also highlights the importance of exchange rate regimes for market integration, particularly in countries with fragmented or parallel currency markets. Negassa and Myers demonstrated how dual exchange rate systems can systematically distort spatial price relationships, creating persistent arbitrage opportunities that fail to be eliminated through normal market mechanisms.\autocite{negassa2007} This finding has direct relevance for Yemen, where the dual exchange rate regime represents a fundamental challenge to market integration.

\subsection{Research Gap and Contribution}

While extensive research exists on market integration in developing countries, and a growing literature addresses conflict-affected economies, few studies comprehensively analyze market fragmentation in settings with both active conflict and dual exchange rate regimes. This paper contributes to the literature by applying threshold cointegration techniques to Yemen's unique context, explicitly modeling how conflict-induced transaction costs and exchange rate disparities affect commodity price transmission across politically fragmented territories.

Furthermore, by simulating the potential impact of exchange rate unification on threshold parameters and adjustment speeds, this research provides empirically grounded policy insights for enhancing market integration in conflict settings. The paper's emphasis on asymmetric adjustment and spatial dependencies also addresses methodological gaps in the existing literature, offering a more nuanced understanding of market dynamics in complex emergency environments.